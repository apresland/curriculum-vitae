%!TEX TS-program = xelatex
%!TEX encoding = UTF-8 Unicode
% Awesome CV LaTeX Template for Cover Letter
%
% This template has been downloaded from:
% https://github.com/posquit0/Awesome-CV
%
% Authors:
% Claud D. Park <posquit0.bj@gmail.com>
% Lars Richter <mail@ayeks.de>
%
% Template license:
% CC BY-SA 4.0 (https://creativecommons.org/licenses/by-sa/4.0/)
%


%-------------------------------------------------------------------------------
% CONFIGURATIONS
%-------------------------------------------------------------------------------
% A4 paper size by default, use 'letterpaper' for US letter
\documentclass[11pt, a4paper]{awesome-cv}

% Configure page margins with geometry
\geometry{left=1.4cm, top=.8cm, right=1.4cm, bottom=1.8cm, footskip=.5cm}

% Specify the location of the included fonts
\fontdir[fonts/]

% Color for highlights
% Awesome Colors: awesome-emerald, awesome-skyblue, awesome-red, awesome-pink, awesome-orange
%                 awesome-nephritis, awesome-concrete, awesome-darknight
\colorlet{awesome}{awesome-red}
% Uncomment if you would like to specify your own color
% \definecolor{awesome}{HTML}{CA63A8}

% Colors for text
% Uncomment if you would like to specify your own color
% \definecolor{darktext}{HTML}{414141}
% \definecolor{text}{HTML}{333333}
% \definecolor{graytext}{HTML}{5D5D5D}
% \definecolor{lighttext}{HTML}{999999}

% Set false if you don't want to highlight section with awesome color
\setbool{acvSectionColorHighlight}{true}

% If you would like to change the social information separator from a pipe (|) to something else
\renewcommand{\acvHeaderSocialSep}{\quad\textbar\quad}


%-------------------------------------------------------------------------------
%	PERSONAL INFORMATION
%	Comment any of the lines below if they are not required
%-------------------------------------------------------------------------------
% Available options: circle|rectangle,edge/noedge,left/right
\photo[circle,noedge,left]{./examples/profile}
\name{Andrew}{Presland}
\position{Machine Learning{\enskip\cdotp\enskip}Data Science{\enskip\cdotp\enskip}Data Engineering}
\address{Oberschlossfeld 33, Willisau, 6130, Switzerland}

\mobile{(+41) 79-960-8333}
\email{contact@presland.io}
\homepage{www.presland.io}
\github{apresland}
\linkedin{apresland}
% \gitlab{gitlab-id}
% \stackoverflow{SO-id}{SO-name}
% \twitter{@twit}
% \skype{skype-id}
% \reddit{reddit-id}
% \medium{madium-id}
% \googlescholar{googlescholar-id}{name-to-display}
%% \firstname and \lastname will be used
% \googlescholar{googlescholar-id}{}
% \extrainfo{extra informations}

%\quote{``Be the change that you want to see in the world."}


%-------------------------------------------------------------------------------
%	LETTER INFORMATION
%	All of the below lines must be filled out
%-------------------------------------------------------------------------------
% The company being applied to
\recipient
  {Tamara Hüsser}
  {Besi Switzerland AG\\Hinterbergstrasse 32a\\6132, Steinhausen}
% The date on the letter, default is the date of compilation
\letterdate{\today}
% The title of the letter
\lettertitle{Job Application for (Sr) Machine Learning and Computer Vision Engineer}
% How the letter is opened
\letteropening{Dear Ms. Hüsser,}
% How the letter is closed
\letterclosing{Sincerely,}
% Any enclosures with the letter
\letterenclosure[Attached]{Curriculum Vitae}


%-------------------------------------------------------------------------------
\begin{document}

% Print the header with above personal informations
% Give optional argument to change alignment(C: center, L: left, R: right)
\makecvheader[R]

% Print the footer with 3 arguments(<left>, <center>, <right>)
% Leave any of these blank if they are not needed
\makecvfooter
  {\today}
  {Andrew Presland~~~·~~~Cover Letter}
  {}

% Print the title with above letter informations
\makelettertitle

%-------------------------------------------------------------------------------
%	LETTER CONTENT
%-------------------------------------------------------------------------------
\begin{cvletter}

\lettersection{About Me}
I am a Machine Learning Engineer with background in embedded systems software 
engineering (C++) and trained research scientist (Physics/PhD).
I am enthusiastic about the application of machine learning to perception especially 
in embedded systems where performance challenges are imposed by constrained resources. 
I have taken a one year self-funded sabbatical specifically designed to enhance 
expertise in deep learning, computer vision and robotics (see CV / blog / github). 

I have been resident in Switzerland for 20 years, I am married to a Swiss
national, hold a C permit and speak german.

\lettersection{Why Besi?}
I worked at Besi for two years as a consultant software engineer (bbv Software Services) 
working in the pick-and-place team and later in the scrum team lead by 
Hans-Georg Kohler. From that experience I know Besi to be an innovative company 
focused on quality engineering where computer vision is crucial. Working at Besi 
would provide the opportunity for me to exploit experience in Embedded Systems, ML and Perception 
and cloud based Data Pipelines in a single role with company wide impact.


\lettersection{Why Me?}

I am passionate about deep learning and computer vision with particular interest
in building real-time perception systems. I am a Certified Tensorflow 
Developer and completed the Tensorflow Advanced Techniques specialism. 
I have experience of accelerating Tensorflow models with NVIDIA GPUs using 
Tensorflow/TensorRT achieving object detection at 40 FPS on a Jetson Nano. I also 
bring knowledge of computer vision having developed a stereo visual odometry solution 
based on OpenCV and ROS.  

From my previous (Data Scientist) post I bring experience in using Python to develop ML models 
with Tensorflow/scikit-learn and ML production pipelines with Tensorflow Extended (TFX), 
Apache Beam and AWS (Lambda/Batch). As a PhD physicist I have a
good grounding in the underpinning of ML (Linear Algebra, Calculus and Probability). 
As an experienced software engineer I have strong skills in Design Patterns, SOLID, TDD, CI/CD and 
Clean Code. Additionally I am also interested in high-performance computing on heterogeneous 
embedded systems and optimising (C++) algorithms using techniques like 
vectorization (e.g. SIMD).


\end{cvletter}


%-------------------------------------------------------------------------------
% Print the signature and enclosures with above letter informations
\makeletterclosing

\end{document}