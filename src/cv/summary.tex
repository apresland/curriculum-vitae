%-------------------------------------------------------------------------------
%	SECTION TITLE
%-------------------------------------------------------------------------------
\cvsection{Summary}


%-------------------------------------------------------------------------------
%	CONTENT
%-------------------------------------------------------------------------------
\begin{cvparagraph}

%-------------------------------------------------------------------------------
Focused on developing data intensive applications with low-latency and real-time requirements, 
I am skilled in developing in Python, C++, Java and Scala. I have recently architected
a biometrics data system for on-premises bare metal Kubernetes clusters that provides 
scalable real-time biometric inference for airports. This employs an edge-to-cloud design
the system synchronizes local object stores and RDBMS with AWS.

I started my career obtaining a PhD in Experimental Particle Physics at CERN developing 
electromagnetic calorimeters based on Avalanche Photodiodes (APD) and 
Vacuum Phototriodes (VPT) and Ring Imaging Cherenkov Detectors based on 
Hybrid Photon Detectors (HPD). In addition to detector prototyping I was 
involved in developing the physics event data model and event reconstruction software. 
I was CERN Research Fellow for applied physics working on Monte Carlo simulation of 
radiation induced degradation to silicon devices in the LHC beamline due to Single Event Errors 
and total ionizing radiation.

I have previously worked on Embedded Systems software engineering developing real-time 
software for constrained systems running RTOS and embedded Linux (motion control, IoT). 
More recently I have been active as a Data Scientist in SmartMobility using 
remote sensing data (GPS/IMU) to reconstruct latent human mobility patterns.

Key interests: intelligent systems, machine perception, state-estimation, 
high performance computing and data systems engineering.
\end{cvparagraph}