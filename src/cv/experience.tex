%-------------------------------------------------------------------------------
%	SECTION TITLE
%-------------------------------------------------------------------------------
\cvsection{Experience}


%-------------------------------------------------------------------------------
%	CONTENT
%-------------------------------------------------------------------------------
\begin{cventries}

%-------------------------------------------------------------------------------
\cventry
{Travizory Border Security SA.} % Organization
{Machine Learning Engineer (Python/C++)} % Job title
{Neuchatel, Switzerland} % Location
{2022 - 2024} % Date(s)
{
  \begin{cvitems} % Description
    \item {
    Architected a real-time video ingestion and video inference platform running on 
    bare-metal Kubernetes as part of an edge-to-cloud Digital Identity Management 
    System. The systems microservices archicture used an high speed ZeroMQ based event bus for data 
    throughput, FastAPI to provide management REST endpoints,and WebSockets to 
    serve a stream of biometric events to frontend clients.}
    \item {
    Implemented offline data processing workflows runing on kubernetes using 
    distributed task queues to synchronize data in AWS 
    to on-premises persistence using MinIO object stores, MariaDB using Celery tasks. 
    The workflows consumed biometric images with metadata, extracted face embeddings and built 
    embedding vector indexes that were used for online face similarity search.}
    \item {
    Developed ingestion pipelines for real-time video from GigE Vision industrial cameras 
    directly into bare-metal Kubernetes clusters using macvlan networking and Multus CNI. 
    The video streams were used for biometric face detection at airport borders achieving 
    100 ms recognition latency at 30 FPS.}
  \end{cvitems}
}

%-------------------------------------------------------------------------------
\cventry
  {AxonVIBE AG.} % Organization
  {Data Scientist (Java/Python)} % Job title
  {Luzern, Switzerland} % Location
  {2018 - 2021} % Date(s)
  {
    \begin{cvitems} % Description
      \item {
          Constructed serverless real-time time-series sensor data ingestion (GPS/IMU) based on AWS Lambda and AWS Batch to 
          provide windowed feature extraction, data-monitoring, anomaly detection using 
          Apache Beam and Tensorflow TFX.}
      \item {
          Developed algorithms and data models for a set of Spring Boot microservices that perform hierarchical 
          spatiotemporal clustering for Significant Location and user trajectory detection for a location based contextual platform
          using machine learning models to infer user behavioral patterns}
      \item {
          Developed ML pipelines for hyper-parameter search, training and evaluating of Neural Networks, Random Forests and Gradient Boosting Machines.}
    \end{cvitems}
  }

%-------------------------------------------------------------------------------
\cventry
{bbv Software Services AG.} % Organization
{Senior Software Engineer (C++/Java)} % Job title
{Luzern, Switzerland} % Location
{2010 - 2018} % Date(s)
{
  \begin{cvitems} % Description
    \item {
        Consultant Engineer collaborating in Agile projects at multinational
        customers. Used Scrum and Kanban to iteratively deliver 
        value. Integrated constantly, refactored ruthlessly, and avoided smells. 
        Favoured SOLID principles and relied on design patterns to produce clean, 
        testable code measured using static analysis.
        }
    \item {
        Developed in modern C++ for embedded Linux and RTOS targets and 
        multiple architectures (ARM/PowerPC) with cross-toolchains. Applications 
        included Fieldbus based realtime distributed controls for industrial robotics, 
        and interfacing with Java Card algorithms running on tamper-resistant smartcards 
        for security modules (SSL/TLS) embedded in Industrial IoT Gateways.}
    \item {
      Quality Assurance Engineering for medical devices and mobile apps
      with functional specification written in Cucumber, 
      test automation with Jenkins CI and Appium. 
      }  
  \end{cvitems}
}

%-------------------------------------------------------------------------------
\cventry
{Hagenbuch Hydraulic Systems AG.} % Organization
{Software Project Leader} % Job title
{Luzern, Switzerland} % Location
{2006 - 2010} % Date(s)
{
  \begin{cvitems} % Description
    \item {
      Development of PID based real-time motion control for 6-DOF parallel 
      kinematic manipulators (Hexapods) using Fieldbus distributed controllers 
      and signals from positional encoders and force sensors.}
    \item {
      Real-time signal processing in the time and frequency domains for 
      vibration analytics systems. Spectral analysis, and development of 
      real-time data visualization applications.}
  \end{cvitems}
}

%-------------------------------------------------------------------------------
\cventry
{European Organization for Nuclear Research (CERN)} % Organization
{Physics Research Fellow (Beamlines)} % Job title
{Geneva, Switzerland} % Location
{2003 - 2006} % Date(s)
{
  \begin{cvitems} % Description
    \item {
      Monte-Carlo simulation of radiation induced Single Event Errors (SSE) and 
      Total Ionizing Dose (TID) in LHC control electronics under 
      beam accident scenarios using intranuclear cascades, multi-particle 
      transport and detailed 3D combinatorial models.}
      \item{
      Scientific data analysis and visualization in Matlab. Presentation of findings
      to the LHC collaboration to inform the LHC design and at international scientific conferences.}
  \end{cvitems}
}

%-------------------------------------------------------------------------------
\cventry
{H. H. Wills Physics Laboratory, University of Bristol} % Organization
{Post Doctoral Research Associate (Particle Physics)} % Job Title
{Geneva, Switzerland} % Location
{2001 - 2003} % Date(s)
{
  \begin{cvitems} % Description
    \item {
      Member of the LHCb collaboration searching for the source of matter-antimatter 
      asymmetry in the universe. I helped develop the Hybrid Photon Detector (HPD) based 
      Ring Imaging Cherenkov Detector (RICH) providing particle Identity (PID) 
      for LHCb physics event reconstruction.}
    \item {
      Developed HPD simulation and physics event reconstruction code (C++) for 
      the LHCb Data Processing Framework and helped define Definition of the 
      LHCb Physics Data Model. Performed Scientific Data Analysis using ROOT Data 
      Analysis Framework for physics reach studies.
        }
  \end{cvitems}
}

%-------------------------------------------------------------------------------
\cventry
{H. H. Wills Physics Laboratory, University of Bristol} % Organization
{Doctoral Researcher (Particle Physics)} % Job Title
{Bristol, UK} % Location
{1998 - 2001} % Date(s)
{
  \begin{cvitems} % Description
    \item {
        Member of the CMS collaboration focussed on the search for the Higgs Boson
        helping develop the Avalanche Photodiode (APD) and Vacuum Phototriode (VPT) 
        based Electromagnetic Calorimetor (ECAL). This is used to measure the energy 
        of photons and electrons in cascades of sub-atomic particles produced by 
        proton collisions.}
      \item {
        Development of algorithms to reconstruct relativistic electron trajectories 
        using Kalman filters and electron energy by photon counting based on Geant4 
        Monte Carlo simulation of LHC physics events.}
  \end{cvitems}
}

%-------------------------------------------------------------------------------
\end{cventries}