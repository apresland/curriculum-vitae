%!TEX TS-program = xelatex
%!TEX encoding = UTF-8 Unicode

%-------------------------------------------------------------------------------
% CONFIGURATIONS
%-------------------------------------------------------------------------------
% A4 paper size by default, use 'letterpaper' for US letter
\documentclass[11pt, a4paper]{awesome-cv}

% Configure page margins with geometry
\geometry{left=1.4cm, top=.8cm, right=1.4cm, bottom=1.8cm, footskip=.5cm}

% Specify the location of the included fonts
\fontdir[fonts/]

% Color for highlights
% Awesome Colors: awesome-emerald, awesome-skyblue, awesome-red, awesome-pink, awesome-orange
%                 awesome-nephritis, awesome-concrete, awesome-darknight
\colorlet{awesome}{awesome-red}

% Set false if you don't want to highlight section with awesome color
\setbool{acvSectionColorHighlight}{true}

% If you would like to change the social information separator from a pipe (|) to something else
\renewcommand{\acvHeaderSocialSep}{\quad\textbar\quad}


%-------------------------------------------------------------------------------
%	PERSONAL INFORMATION
%-------------------------------------------------------------------------------
% Available options: circle|rectangle,edge/noedge,left/right
% \photo{./src/profile.png}
\name{Andrew}{Presland}
\position{Software Engineer{\enskip\cdotp\enskip}Machine Learning}
\address{Oberschlossfeld 33, Willisau, 6130, Switzerland}

\mobile{(+41) 79-960-8333}
\email{andrew.presland@gmail.com}
\homepage{www.presland.io}
\github{apresland}
\linkedin{apresland}

\quote{``Be the change that you want to see in the world."}


%-------------------------------------------------------------------------------
\begin{document}

% Print the header with above personal informations
% Give optional argument to change alignment(C: center, L: left, R: right)
\makecvheader

% Print the footer with 3 arguments(<left>, <center>, <right>)
% Leave any of these blank if they are not needed
\makecvfooter
  {\today}
  {Andrew Presland~~~·~~~Curriculum Vitae}
  {\thepage}


%-------------------------------------------------------------------------------
%	CV/RESUME CONTENT
%	Each section is imported separately, open each file in turn to modify content
%-------------------------------------------------------------------------------
%-------------------------------------------------------------------------------
%	SECTION TITLE
%-------------------------------------------------------------------------------
\cvsection{Experience}


%-------------------------------------------------------------------------------
%	CONTENT
%-------------------------------------------------------------------------------
\begin{cventries}

%-------------------------------------------------------------------------------
  \cventry
    {AxonVIBE AG.} % Organization
    {Data Scientist (Python / Java)} % Job title
    {Luzern, Switzerland} % Location
    {2018 - 2021} % Date(s)
    {
      \begin{cvitems} % Description
        \item {
            Detected behavioural patterns for a location based contextual platform 
            by builing machine learning models from mobile sensor time series data 
            (GNSS/IMU) and deploying on AWS.}
        \item {
            Trained, and deployed Neural Networks, Gradient Boosting Machines, Spatio-Temporal 
            Clustering and Markov Models.}
        \item {
            Constructed serverless injestion for data-monitoring, anomaly detection, 
            and feature extraction (Apache Beam, AWS Lambda, Batch and S3).}
        \item {
            Developed hierarchical clustering algorithms for Significant Location Detection 
            and productionized them
            as a backend service in Java with ETL, SQL data connectors and Spring Boot REST interface.}
      \end{cvitems}
    }

%-------------------------------------------------------------------------------
\cventry
{bbv Software Services AG.} % Organization
{Senior Software Engineer (C++ / Java)} % Job title
{Luzern, Switzerland} % Location
{2010 - 2018} % Date(s)
{
  \begin{cvitems} % Description
    \item {
        Consultant engineer developing software solutions using 
        Agile methods (Scrum, TDD, CleanCode, Static Analysis).}
      \item{
        Projects included distributed real-time 
        controls, industrial robotics, communication and security modules 
        for IoT and SmartEnergy metering platform.}
    \item {
        Quality Assurance Engineering for medical devices and mobile apps
        using functional specification written in Cucumber, test automation with Appium
        and Jenkins CI, configuration management with JFrog Artifactory. 
        }
  \end{cvitems}
}

%-------------------------------------------------------------------------------
\cventry
{Hagenbuch Hydraulic Systems AG.} % Organization
{Project Leader (Software)} % Job title
{Luzern, Switzerland} % Location
{2006 - 2010} % Date(s)
{
  \begin{cvitems} % Description
    \item {
        Development of real-time motion control systems for 6DOF parallel 
        kinematic manipulators (Stewart Platforms) and Vibration Analysis rigs,
        real-time signal processing, spectral analysis, and HMI development (.Net).}
    \item {
        Project comissioning (on-site) and Level-2 technical support.}
  \end{cvitems}
}

%-------------------------------------------------------------------------------
\cventry
{European Organization for Nuclear Research (CERN)} % Organization
{Research Fellow (Applied Physics)} % Job title
{Geneva, Switzerland} % Location
{2003 - 2006} % Date(s)
{
  \begin{cvitems} % Description
    \item {
        Computational Physics and 3D geometric modeling for Accelerator Driven Systems (ADS) using the CERN 
        high performance computing cluster.}
    \item {
        Monte-Carlo simulation of multi-particle transport for energy deposition 
        and radiation damage in the LHC control electronics under 
        beam accident scenarios.} 
        \item{
        Data analysis and visualization of scientific results in Matlab.}
  \end{cvitems}
}

%-------------------------------------------------------------------------------
\cventry
{H. H. Wills Physics Laboratory, University of Bristol} % Organization
{Post Doctoral Research Associate (Experimental Particle Physics)} % Job Title
{Geneva, Switzerland} % Location
{2001 - 2003} % Date(s)
{
  \begin{cvitems} % Description
    \item {
        Member of the LHCb collaboration searching for the source of matter-antimatter 
        asymmetry in the universe by studying decays of sub-atomic particles containing b-quarks.}
    \item {
        Definition and imlimentation of the physics Object Data Model in C++}
    \item {
        Development C++ code 
        for the Data Processing Framework (Gaudi) and performed Monte-Carlo simulation
        and Scientific Data Analysis using ROOT Data Analysis Framework.}
  \end{cvitems}
}

%-------------------------------------------------------------------------------
\cventry
{H. H. Wills Physics Laboratory, University of Bristol} % Organization
{Doctoral Researcher (Experimental Particle Physics)} % Job Title
{Bristol, UK} % Location
{1998 - 2001} % Date(s)
{
  \begin{cvitems} % Description
    \item {
        Member of the CMS collaboration, a general purpose detector at the LHC primarily
        focussed on the search for the Higgs Boson.}
    \item {
        Scientific data analysis using ROOT Data Analysis Framework and Geant4 Monte Carlo simulation of physics processes in C++.}
    \item {
      Subatomic particle detection and tracking using Kalman filters, energy clustering.
    }
  \end{cvitems}
}

%-------------------------------------------------------------------------------
\end{cventries}


%-------------------------------------------------------------------------------
\end{document}