%!TEX TS-program = xelatex
%!TEX encoding = UTF-8 Unicode

%-------------------------------------------------------------------------------
% CONFIGURATIONS
%-------------------------------------------------------------------------------
% A4 paper size by default, use 'letterpaper' for US letter
\documentclass[11pt, a4paper]{awesome-cv}

% Configure page margins with geometry
\geometry{left=1.4cm, top=.8cm, right=1.4cm, bottom=1.8cm, footskip=.5cm}

% Specify the location of the included fonts
\fontdir[fonts/]

% Color for highlights
% Awesome Colors: awesome-emerald, awesome-skyblue, awesome-red, awesome-pink, awesome-orange
%                 awesome-nephritis, awesome-concrete, awesome-darknight
\colorlet{awesome}{awesome-red}

% Set false if you don't want to highlight section with awesome color
\setbool{acvSectionColorHighlight}{true}

% If you would like to change the social information separator from a pipe (|) to something else
\renewcommand{\acvHeaderSocialSep}{\quad\textbar\quad}


%-------------------------------------------------------------------------------
%	PERSONAL INFORMATION
%-------------------------------------------------------------------------------
% Available options: circle|rectangle,edge/noedge,left/right
% \photo{./src/profile.png}
\name{Andrew}{Presland}
\position{Software Engineer{\enskip\cdotp\enskip}Machine Learning}
\address{Oberschlossfeld 33, Willisau, 6130, Switzerland}

\mobile{(+41) 79-960-8333}
\email{andrew.presland@gmail.com}
\homepage{www.presland.io}
\github{apresland}
\linkedin{apresland}

\quote{``Be the change that you want to see in the world."}


%-------------------------------------------------------------------------------
\begin{document}

% Print the header with above personal informations
% Give optional argument to change alignment(C: center, L: left, R: right)
\makecvheader

% Print the footer with 3 arguments(<left>, <center>, <right>)
% Leave any of these blank if they are not needed
\makecvfooter
  {\today}
  {Andrew Presland~~~·~~~Curriculum Vitae}
  {\thepage}


%-------------------------------------------------------------------------------
%	CV/RESUME CONTENT
%	Each section is imported separately, open each file in turn to modify content
%-------------------------------------------------------------------------------
%-------------------------------------------------------------------------------
%	SECTION TITLE
%-------------------------------------------------------------------------------
\cvsection{Experience}


%-------------------------------------------------------------------------------
%	CONTENT
%-------------------------------------------------------------------------------
\begin{cventries}

%-------------------------------------------------------------------------------
  \cventry
    {AxonVIBE AG.} % Organization
    {Data Scientist (Location Technology)} % Job title
    {Luzern, Switzerland} % Location
    {2018 - 2021} % Date(s)
    {
      \begin{cvitems} % Description
        \item {
            Data Scientist in a team responsible for detection and classification 
            of behavioural patterns in users mobility starting from mobile sensor data (GNSS/IMU).}
        \item {
            Pre-processing, Anomaly Detection and Monitoring of remote sensor data 
            using Apache Beam, TensorFlow TFX Transform , TFX Data Validation.}
        \item {
            Explorative Data Analysis using in Python with Scikit-learn, matplotlib and Jupyter Notebook.}
        \item {
            Development of Machine Learning models (Random Forest, Gradient 
            Boosting Machine) involving feature engineering, feature selection, hyperparameter search, cross validation 
            and model evaluation with confusion matrices, RoC curves for trajectory classification.}
        \item {
            Development of Deep Neural Networks for deployment on edge (mobile) devices for
            Human Activity Detection based on data from IMU sensors.}
        \item {
            Development of graphical models aggregating user data based as descrete-time
            2nd order Markov Chains to predict users future destinations and mode-of-transport.}
        \item {
            Developed a Significant Location Detection process in Python using spatio-temporal clustering to 
            segment GPS trajectories into stop/move episodes, density based clustering of 
            stop episodes into significant locations with rules based semantic labeling.}
        \item {
            Productionized Significant Location Detection in Java as a 
            Spring Boot Batch service including ETL and JDBC data connections.}
        \item {
            Exposed significant locations as a Spring Boot RESTful service and 
            visualization with Javascript frontend dashboard.}
      \end{cvitems}
    }

%-------------------------------------------------------------------------------
\cventry
{bbv Software Services AG.} % Organization
{Senior Software Engineer (Embedded Systems)} % Job title
{Luzern, Switzerland} % Location
{2010 - 2018} % Date(s)
{
  \begin{cvitems} % Description
    \item {
        Consultant Software Engineer with focus on delivering high quality 
        testable embedded software by applying Agile, Extreme Programming and SOLID
        design principles.}
    \item {
        Development of the Communication (TCP/IP) and Security (SSL/TLS) modules of
        a SmartEnergy IoT Gateway based on a STM32 Arm Cortex embedded system running Segger the embOS RTOS.}
    \item {
        Development of the Hardware Abstraction Layer for a SmartEnergy platform 
        based on a STM32 Arm Cortex embedded systems and running the ThreadX RTOS.}
    \item {
        Development of calibration and real-time motion control software for
        industrial robots used in semiconductor fabrication. The master-slave 
        real-time distributed control system communicated over fieldbus on a PowerPC embedded system
        running Indel INOS RTOS.
        }
    \item {
        Quality Assurance Engineering for a medical system consisting of a mobile application 
        providing diabetes management of an autonomous insulin pump controled over bluetooth. 
        Test automation consisting of device abstractionin XML and Java,  executable behaviour specification with Cucumber, 
        artifact managment with JFrog and test with Appium and CI with Jenkins. 
        }
  \end{cvitems}
}

%-------------------------------------------------------------------------------
\cventry
{Hagenbuch Hydraulic Systems AG.} % Organization
{Project Leader (Software Engineering)} % Job title
{Luzern, Switzerland} % Location
{2006 - 2010} % Date(s)
{
  \begin{cvitems} % Description
    \item {
        Development of real-time motion control systems (PID) for 6DOF parallel 
        kinematic manipulators (Stewart Platforms) and high performance shakers 
        (Vibration Analysis) using distributed systems connected by fieldbus.}
    \item {
        Real-time signal processing for motion control optimization 
        (harmonic suppression) and visualization (spectral analysis).}
    \item {
        HMI application development using the Microsoft .Net framework.}
    \item {
        Level-2 technical support.}
  \end{cvitems}
}

%-------------------------------------------------------------------------------
\cventry
{European Organization for Nuclear Research (CERN)} % Organization
{Research Fellow (Applied Physics)} % Job title
{Geneva, Switzerland} % Location
{2003 - 2006} % Date(s)
{
  \begin{cvitems} % Description
    \item {
        Computational Physics developer for the Emerging Energy Technology section 
        working on Accelerator Driven Systems (ADS) by applying Monte-Carlo particle 
        transport simulation to Combinatorial Geometry models of the LHC.}
    \item {
        Monte-Carlo simulation of multi-particle transport for studies of thermal 
        energy deposition via intra-nuclear cascade within the LHC beam-line magnets 
        under accident scenarios at the LHC.
    \item {
        Studies of the single-event upset rate in the LHC control electronics under 
        beam accident scenarios which can result in degredation of LHC controls}
    \item {
        Application of the CERN high performance computing cluster to Monte-Carlo 
        particle transport simulation.}
    }
  \end{cvitems}
}

%-------------------------------------------------------------------------------
\cventry
{H. H. Wills Physics Laboratory, University of Bristol} % Organization
{Post Doctoral Research Associate (High Energy Physics)} % Job Title
{Geneva, Switzerland} % Location
{2001 - 2003} % Date(s)
{
  \begin{cvitems} % Description
    \item {
        Member of the LHCb collaboration, a detector at the LHC searching for the source of
        matter-antimatter asymmetry in the universe by studying particles containing b-quarks.}
    \item {
        Contributed to the Object Data Model for the LHCb Event Data and development of the 
        LHCb Event Data Processing Framework (Gaudi)}
    \item {
        Scientific Data Analysis of simulated (Monte-Carlo) event data using ROOT Data Analysis Framework.
        }
  \end{cvitems}
}

%-------------------------------------------------------------------------------
\cventry
{H. H. Wills Physics Laboratory, University of Bristol} % Organization
{Doctoral Researcher (High Energy Physics)} % Job Title
{Bristol, UK} % Location
{1998 - 2001} % Date(s)
{
  \begin{cvitems} % Description
    \item {
        Member of the CMS collaboration, a general purpose detector at the LHC primarily
        focussed on the search for the Higgs Boson.}
    \item {
        Prototyping of the PbWO4 scintillating crystal calorimeters and photon detector readout for 
        detecting electromagnetic particels.
        }
    \item {
        Reconstruction of particle energy using clustering algorithms to process signals from photon sensor readout.
    }
  \end{cvitems}
}

%-------------------------------------------------------------------------------
\end{cventries}


%-------------------------------------------------------------------------------
\end{document}